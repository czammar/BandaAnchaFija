\documentclass{article}
\usepackage{amsmath}
\usepackage{amsthm}
\usepackage{amsfonts}
\usepackage{hyperref}
\usepackage{graphicx}
\graphicspath{ {images/} }
\usepackage[usenames, dvipsnames]{color}
\usepackage[T1]{fontenc}
\usepackage[spanish, activeacute]{babel}
\usepackage[utf8]{inputenc} 
\usepackage{array}
\usepackage{float}
\usepackage{amssymb}
\usepackage{listings}
\usepackage[inline]{enumitem}
\usepackage{color} %red, green, blue, yellow, cyan, magenta, black, white
\definecolor{mygreen}{RGB}{28,172,0} % color values Red, Green, Blue
\definecolor{mylilas}{RGB}{170,55,241}
\usepackage[]{algorithm2e}
\usepackage{geometry}
\setlength{\topmargin}{-5em} %
\setlength{\headsep}{15pt} %
\setlength{\footskip}{25pt} %
\setlength{\textheight}{9.75in} %
\setlength{\textwidth}{6.5in} %
\setlength{\hoffset}{-15pt}

\usepackage{marvosym}
\makeatletter
\newcommand{\distas}[1]{\mathbin{\overset{#1}{\kern\z@\sim}}}%
\newsavebox{\mybox}\newsavebox{\mysim}
\newcommand{\distras}[1]{%
  \savebox{\mybox}{\hbox{\kern3pt$\scriptstyle#1$\kern3pt}}%
  \savebox{\mysim}{\hbox{$\sim$}}%
  \mathbin{\overset{#1}{\kern\z@\resizebox{\wd\mybox}{\ht\mysim}{$\sim$}}}%
}
\title{}

\author{Cesar Zamora}


\lstset{language=MATLAB,%
    %basicstyle=\color{red},
    breaklines=true,%
    morekeywords={matlab2tikz},
    keywordstyle=\color{blue},%
    morekeywords=[2]{1}, keywordstyle=[2]{\color{black}},
    identifierstyle=\color{black},%
    stringstyle=\color{mylilas},
    commentstyle=\color{mygreen},%
    showstringspaces=false,%without this there will be a symbol in the places where there is a space
    numbers=left,%
    numberstyle={\tiny \color{black}},% size of the numbers
    numbersep=9pt, % this defines how far the numbers are from the text
    emph=[1]{for,end,break},emphstyle=[1]\color{red}, %some words to emphasise
    %emph=[2]{word1,word2}, emphstyle=[2]{style},    
}

\newcommand{\norm}[1]{\left\lVert#1\right\rVert}

\newcommand{\normvec}[1]{\left\lVert|#1|\right\rVert}

\theoremstyle{plain}
\newtheorem*{defini*}{Definición}
\newtheorem*{lem*}{Lema}
\newtheorem*{theo*}{Teorema}
\newtheorem*{prop*}{Proposición}
\newtheorem*{cor*}{Corolario}

\newcounter{problem}
\newcounter{solution}

\newcommand\Problem{%
  \stepcounter{problem}%
  \textbf{\theproblem.}~%
  \setcounter{solution}{0}%
}

\newcommand\TheSolution{%
  \textbf{Solución:}\\%
}

\newcommand\ASolution{%
  \stepcounter{solution}%
  \textbf{Solution \TheSolution:}\\%
}
\parindent 0in
\parskip 1em


\begin{document}

\section*{Apéndice: Códigos de R}

A continuación se presentan los códigos utilizados en este proyecto.

\subsection*{Archivo creating\_conapo.R}

\begin{lstlisting}[language=R]
library(readr)
library(tidyverse)

######---- Carga base de datos de accesos de CONAPO ----######
conapo <- read_csv("CONAPO/Base_Indice_de_marginacion_municipal_90-15.csv", col_types = cols(CVE_ENT = col_character(), 
CVE_MUN = col_character()), locale = locale(encoding = "ISO-8859-1"))
# Creamos variables de id de entidad y municipio
conapo$K_ENTIDAD<-NA
conapo$K_MUNICIPIO<-NA

for (index in 1:nrow(conapo)){
conapo$K_ENTIDAD[index] = ifelse(nchar(conapo$CVE_ENT[index])==1, paste(0,conapo$CVE_ENT[index],sep=""),conapo$CVE_ENT[index])
}

for (index in 1:nrow(conapo)){
conapo$K_MUNICIPIO[index] = ifelse(nchar(conapo$CVE_MUN[index])==4, substr(conapo$CVE_MUN[index],2,4),substr(conapo$CVE_MUN[index],3,5))
}

conapo<- conapo %>% mutate(K_ENTIDAD_MUNICIPIO = paste(K_ENTIDAD, K_MUNICIPIO,sep="")) 
conapo <- subset(conapo, ENT != "Nacional")

# Escribe la base de datos de conapo
write_csv(conapo, "CONAPO_2015.csv")

\end{lstlisting}


\begin{lstlisting}[language=Bash]
 cd
\end{lstlisting}

\begin{lstlisting}[language=Python]
print("hola")
\end{lstlisting}

\end{document}
